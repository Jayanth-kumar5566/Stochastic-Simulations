\documentclass[14pt]{article}
\usepackage{fullpage}
\usepackage{color}
\usepackage[affil-it]{authblk}
\begin{document}
\author{Jayanth Kumar N}
\affil{Indian Institute of Science Education and Research, Pune}
\title{\textbf{Readings on the papers}}
\maketitle

\section{What I need to do?}
To create a SIR model of two cities with different $R_{0}$'s connected
by a transport and figure out the $R_{0}$ of the new connected system.

\begin{description}
\item[From readings of paper I, Day I] \hfill \\ The calculations of
  individual city $R_{0}$ depends on the SIR model of the paticular
  city, ie, It does not account for the transportation term or the
  city is considered to be a closed system.
\item[Intrinsic $R_{0}$] \hfill \\ But the $R_{0}$ depends only on the
  parameters $\alpha$ and $\beta$ which are dependent only on the
  disease. So, How come the two cities can have different $R_{0}$ for
  the same disease.
  
  \\ { \color{blue} Todo: Figure out $R_{0}$ is defined and how do we
    calculate it for a model.}  \\ $R_{0}$ definition is the number of
  secondary infections caused a single infected individual.  Since we
  have two cities the $R_{0}$ depends only on $\gamma$ ,
  $\beta_{1)$,$\beta_{2}$ and $N$ the total size of the population.

  
\end{description}

\section{Modeling the Worldwide Spread of Pandemic
Influenza: Baseline Case and Containment
Interventions}
\begin{description}
  \item[Meta population stochastic epidemic model] \hfill \\ This also
    considers the airline flow travel.
  \item[Network] Vertices are Cities and weighted edges represent the
    rate of passenger flow. Each vertice with a weight representing
    the population.
  \item[The Model] \hfill \\ We use \textbf{SLIR} model.
  \item[The Stochastic transport operator] \hfill \\ $\omega$ acts as
    a coupling term between the equations.{\color{blue} To read about}
  \item[Langevin formulation] {\color{blue} Todo}
  \item[Discrete nature of Individuals] \hfill \\ They have used a
    special type of calculation to account for the discreteness of the
    Individuals.{\color{blue} To see Text S1 and figure out}
\end{description}
 {\color{blue} Since $R_{0}$ is function of the disease
   parameters, technically the $R_{0}$'s of all the cities must be the
   same. Since $R_{0}$ doesn't depend on $\omega$. Here in this paper
   we have different $R_{0}$'s due to seasonality changes.ie, their
   transmission coefficient $\beta$ is different due to seasons.Should
   we also consider giving different $\beta$ values to different
   cities to create cities with different $\R_{0}$'s?}

 \\ Yes, techincally we should consider different $\beta$ values for different
 cities since it denotes the rate of contacts inside the city and
 connectivity of the poeple may be different in different cities. But
 the $\gamma$ value of the cities must be the same. Since it doesnot
 depend the location or the spatial distribution of the poeple. Hence we get different $R_{0}$
 \section{Devolopments}
 The initial exponential rise in propotion of susceptible who are
 infected $p(t)$ depends on the rate of which new infections are being
 produced at the rate $\lambda$
 $$P(t)=P(0)e^{\lambda t}$$
 $$\lambda = \frac{R_{0}-1}{\gamma}$$ We use $R_{0}=\frac{\beta
   N}{\gamma}$ as the theoritical value for the cities considering
 simple SIR model.
 \begin{description}
 \item[fitting the slope] Different types of fitting(see note)
 \end{description}
\end{document}
